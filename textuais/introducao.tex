%==============================================================================
\chapter{Introdução}\label{introducao}
%==============================================================================

Este trabalho mostra alguns exemplos de utilização de comandos \LaTeX, opções de formatação e dicas de conteúdo.
  Várias partes foram retiradas do manual da classe \abnTeX~\cite{abntex2classe,abntex2cite}, e algumas partes, principalmente o anexo \ref{anexo:latex}, de \cite{Moro2012}.
  A propósito, recomenda-se a leitura de \cite{Moro2012}, pois contém dicas de como escrever um trabalho de pós-graduação que podem ser aplicadas também a \acp{tcc}.
  Leia também \cite{SisbiUnipampa2011} para mais informações sobre como escrever um trabalho.


%------------------------------------------------------------------------------
\section{Divisões do documento: seção}\label{sec:divisoes}
%------------------------------------------------------------------------------

Esta seção testa o uso de divisões de documentos. Isto é uma seção.


%------------------------------------------------------------------------------
\subsection{Divisões do documento: subseção}
%------------------------------------------------------------------------------

Isto é uma subseção.


%------------------------------------------------------------------------------
\subsubsection{Divisões do documento: subsubseção}
%------------------------------------------------------------------------------

Isto é uma subsubseção.


%------------------------------------------------------------------------------
\subsubsection{Divisões do documento: subsubseção}
%------------------------------------------------------------------------------

Isto é outra subsubseção.


%------------------------------------------------------------------------------
\subsection{Divisões do documento: subseção}\label{sec:exemplo-subsec}
%------------------------------------------------------------------------------

Isto é uma subseção.


%------------------------------------------------------------------------------
\subsubsection{Divisões do documento: subsubseção}
%------------------------------------------------------------------------------

Isto é mais uma subsubseção da \autoref{sec:exemplo-subsec}.


%------------------------------------------------------------------------------
\section{Este é um exemplo de nome de seção longo. Ele deve estar alinhado à esquerda e a segunda e demais linhas devem iniciar logo abaixo da primeira palavra da primeira linha}
%------------------------------------------------------------------------------

Isso atende à norma \citeonline[seções de 5.2.2 a 5.2.4]{NBR14724:2011} e \citeonline[seções de 3.1 a 3.8]{NBR6024:2012}.


%------------------------------------------------------------------------------
\section{Consulte o manual da classe \textsf{abntex2}}
%------------------------------------------------------------------------------

Consulte o manual da classe \textsf{abntex2} \cite{abntex2classe} para uma referência completa das macros e ambientes disponíveis.
  Além disso, o manual possui informações adicionais sobre as normas ABNT observadas pelo \abnTeX.


%------------------------------------------------------------------------------
\section{Organização deste trabalho}
%------------------------------------------------------------------------------

No \autoref{desenvolvimento} há várias instruções e dicas de uso deste modelo, e o \autoref{anexo:latex}\todo{BUG do abntex2: deveria referenciar como Anexo.}~traz dicas sobre o uso do \LaTeX.
